\documentclass[a4paper,11pt]{article}
\usepackage{amsmath,amssymb,amsthm, tikz,titlesec,hyperref,esint,braket, graphicx}

\usepackage[a4paper,margin=2cm]{geometry}
\linespread{1.3}
\newtheorem{claim}{Claim}[section]

\newcommand\name{Park Chanwoo}   % Name of the student
\newcommand\university{KAIST} % Name of the university
\newcommand\department{Physics} % Name of the department
\newcommand\studentid{20230297} % Student ID
\newcommand\s{\,\;}


\newenvironment{solution}[1]
  {\renewcommand\qedsymbol{$\square$}\begin{proof}[\textbf{Solution#1}]}
  {\end{proof}}
\newenvironment{note}
  {\renewcommand\qedsymbol{$\blacksquare$}\begin{proof}[\textnormal{\textbf{note}}]}
  {\end{proof}}

\title{KAIST\\2025 PH361 Solid State Physics I\\
Homework 3\bigskip}
\author{\textbf{\Large \name} \\
% University: \university\\
Department: \department\\
Student ID: \studentid}
\date{\today}

\begin{document}
\thispagestyle{empty}
\maketitle
\tableofcontents
\titleformat{\section}[frame]{\pagebreak}{\filright
\footnotesize  \enspace \textsf{KAIST --- PH361 Solid State Physics I 2025 Spring}\enspace}{6pt}{\Large\bfseries\filcenter}

\newcommand{\boltz}{k_{\mathrm{B}}}
\newcommand{\tr}{\operatorname{tr}}
\newcommand{\Li}{\operatorname{Li}}
\newcommand{\hc}{\text{h.c.}}
\newcommand{\cc}{\text{c.c.}}
\newcommand{\floor}{\operatorname{floor}}
\newcommand{\floor}{\operatorname{floor}}

\section{13.1 Reciprocal Lattice}

By definition, the FCC lattice with a conventional cube-edge $a$ is
\begin{equation}
    \mathcal L_{\mathrm{FCC}}(a_1, a_2, a_3)=\{n_1a_1\mathbf e_1+n_2a_2\mathbf e_2 + n_3a_3\mathbf e_3+\mathbf b:n_1, n_2, n_3\in\mathbb Z, \mathbf b\in\mathcal B\}
\end{equation}
Where
\begin{equation}
    \mathcal B=\{(0, 0, 0), (0, a_2/2, a_3/2), (a_1/2, 0, a_3/2), (a_1/2, a_2/2, 0)\}.
\end{equation}
Let's introducing the vectors,
\begin{align}
    \mathbf a_1&=(0, a_2/2, a_3/2),\\ 
    \mathbf a_2&=(a_1/2, 0, a_3/2),\\ 
    \mathbf a_3&=(a_1/2, a_2/2, 0)
\end{align}
Let
\begin{equation}
    \mathcal M=\{n_1\mathbf a_1+n_2\mathbf a_2 + n_3\mathbf a_3:n_1, n_2, n_3\in\mathbb Z\}.
\end{equation}
Since
\begin{align}
    a_1\mathbf e_1=\mathbf a_2 + \mathbf a_3 - \mathbf a_1, \\
    a_2\mathbf e_2=\mathbf a_3 + \mathbf a_1 - \mathbf a_2, \\
    a_3\mathbf e_3=\mathbf a_1 + \mathbf a_2 - \mathbf a_3,
\end{align}
every point in $\mathcal L_{\mathrm{FCC}}(a_1, a_2, a_3)$ can be represented by $n_1\mathbf a_1+n_2\mathbf a_2 + n_3\mathbf a_3$. Hence,
\begin{equation}
    \mathcal L_{\mathrm{FCC}}(a_1, a_2, a_3)\subseteq \mathcal M
\end{equation}
Also, 
\begin{equation}
    n_1\mathbf a_1+n_2\mathbf a_2 + n_3\mathbf a_3=\frac{n_2 + n_3}{2}a_1\mathbf e_1+\frac{n_3 + n_1}{2}a_2\mathbf e_2 + \frac{n_1 + n_2}{2}a_3\mathbf e_3
\end{equation}
and
\begin{equation}
    (n_2 + n_3) + (n_3 + n_1) + (n_1 + n_2) = 2(n_1+n_2+n_3)
\end{equation}
Since their sum is even, the integers $(n_2 + n_3), (n_3 + n_1), (n_1 + n_2)$ are all even or only two of them are odd. Therefore, there exist integers $m_1, m_2, m_3$ and $\mathbf b\in \{0, \mathbf a_1, \mathbf a_2, \mathbf a_3\}$, such that
\begin{equation}
    n_1\mathbf a_1+n_2\mathbf a_2 + n_3\mathbf a_3=a(m_1\mathbf e_1+m_2\mathbf e_2 + m_3\mathbf e_3) + \mathbf b.
\end{equation}
Thus, $\mathcal M\subseteq \mathcal L_{\mathrm{FCC}}(a_1, a_2, a_3)$ and they are equal
\begin{equation}
    \mathcal M = \mathcal L_{\mathrm{FCC}}(a_1, a_2, a_3)
\end{equation}
and $\mathbf a_1, \mathbf a_2, \mathbf a_3$ are a set of primitive lattice vectors of $\mathcal L_{\mathrm{FCC}}(a_1, a_2, a_3)$.

Let's consider the BCC lattice with a conventional cube-edge $a$:
\begin{equation}
    \mathcal L_{\mathrm{BCC}}(a_1, a_2, a_3) = \{n_1a_1\mathbf e_1+n_2a_2\mathbf e_2+n_3a_3\mathbf e_3 + \mathbf b:n_1, n_2, n_3\in \mathbb Z, \mathbf b\in\mathcal B\}
\end{equation}
Where
\begin{equation}
    \mathcal B=\{(0, 0, 0), (a_1/2, a_2/2, a_3/2)\}.
\end{equation}
Introduce the vectors,
\begin{align}
    \mathbf a_1&=(-a_1/2, a_2/2, a_3/2)\\
    \mathbf a_2&=(a_1/2, -a_2/2, a_3/2)\\
    \mathbf a_3&=(a_1/2, a_2/2, -a_3/2).
\end{align}
Then,
\begin{gather}
    a_1\mathbf e_1 = \mathbf a_2 + \mathbf a_3\\
    a_2\mathbf e_2 = \mathbf a_3 + \mathbf a_1\\
    a_3\mathbf e_3 = \mathbf a_1 + \mathbf a_2\\
    (a_1/2, a_2/2, a_3/2) = \mathbf a_1 + \mathbf a_2 + \mathbf a_3
\end{gather}
Hence,
\begin{equation}
    \mathcal L_{\mathrm{BCC}}(a_1, a_2, a_3)\subseteq \{n_1\mathbf a_1+n_2\mathbf a_2+n_3\mathbf a_3:n_1, n_2, n_3\in\mathbb Z\}.
\end{equation}
Conversely,
\begin{equation}
    n_1\mathbf a_1+n_2\mathbf a_2+n_3\mathbf a_3
    =\frac{n_2 + n_3 - n_1}{2}a\mathbf e_1+\frac{n_3 + n_1 - n_2}{2}a\mathbf e_2 + \frac{n_1 + n_2 - n_3}{2}a\mathbf e_3
\end{equation}
Let $n_1 + n_2 + n_3 = 2N + p$. where $N$ is an integer, and $p=0$ or $p=1$. Then,
\begin{align}
    n_1\mathbf a_1+n_2\mathbf a_2+n_3\mathbf a_3
    &=\frac{n_2 + n_3 - n_1}{2}a_1\mathbf e_1+\frac{n_3 + n_1 - n_2}{2}a_2\mathbf e_2 + \frac{n_1 + n_2 - n_3}{2}a_3\mathbf e_3\\
    &=(N - n_1)a_1\mathbf e_1+(N - n_2)a_2\mathbf e_2 + (N - n_3)a_3\mathbf e_3 + p(a_1/2, a_2/2, a_3/2).
\end{align}
Hence,
\begin{equation}
    \{n_1\mathbf a_1+n_2\mathbf a_2+n_3\mathbf a_3:n_1, n_2, n_3\in\mathbb Z\}\subseteq\mathcal L_{\mathrm{BCC}}(a_1, a_2, a_3).
\end{equation}
Therefore,
\begin{equation}
    \mathcal L_{\mathrm{BCC}}(a_1, a_2, a_3) = \{n_1\mathbf a_1+n_2\mathbf a_2+n_3\mathbf a_3:n_1, n_2, n_3\in\mathbb Z\}.
\end{equation}
and $\mathbf a_1, \mathbf a_2, \mathbf a_3$ is a set of primitive lattice vectors of $\mathcal L_{\mathrm{BCC}}(a_1, a_2, a_3)$.

Based on the above facts, let's find the reciprocal lattice of $\mathcal L_{\mathrm{FCC}}(a_1, a_2, a_3)$.

Let $\mathbf b_1, \mathbf b_2, \mathbf b_3$ be the vectors defined by
\begin{align}
    \mathbf b_1
    = \frac{2\pi (\mathbf a_2\times\mathbf a_3)}{\mathbf a_1\cdot(\mathbf a_2\times\mathbf a_3)}
    = \frac{4\pi}{a_1}(-1/2, 1/2, 1/2)\\
    \mathbf b_2
    = \frac{2\pi (\mathbf a_3\times\mathbf a_1)}{\mathbf a_2\cdot(\mathbf a_3\times\mathbf a_1)}
    = \frac{4\pi}{a_2}(1/2, -1/2, 1/2)\\
    \mathbf b_3
    = \frac{2\pi (\mathbf a_1\times\mathbf a_2)}{\mathbf a_3\cdot(\mathbf a_1\times\mathbf a_2)}
    = \frac{4\pi}{a_3}(1/2, 1/2, -1/2)
\end{align}
Where $\mathbf a_1, \mathbf a_2,\mathbf a_3$ is the set of primitive lattice vectors of $\mathcal L_{\mathrm{FCC}}(a_1, a_2, a_3)$ defined above.

Then $\mathbf a_i \cdot\mathbf b_j=2\pi \delta_{ij}$ . Because
\begin{equation}
    (n_1\mathbf a_1+n_2\mathbf a_2 + n_3\mathbf a_3) \cdot (m_1\mathbf b_1+m_2\mathbf b_2+m_3\mathbf b_3)=2\pi(n_1m_1+n_2m_2+n_3m_3)
\end{equation}
and
\begin{equation}
    e^{i(n_1\mathbf a_1+n_2\mathbf a_2 + n_3\mathbf a_3) \cdot (m_1\mathbf b_1+m_2\mathbf b_2+m_3\mathbf b_3)}=e^{2\pi i(n_1m_1+n_2m_2+n_3m_3)}=1.
\end{equation}
Therefore, the reciprocal lattice $\mathcal L_{\mathrm{FCC}}(a_1, a_2, a_3)$ of $\mathcal L_{\mathrm{FCC}}(a_1, a_2, a_3)$ is
\begin{equation}
    \mathcal L_{\mathrm{FCC}}^*(a_1, a_2, a_3)=\{n_1\mathbf b_1+n_2\mathbf b_2+n_3\mathbf b_3:n_1, n_2, n_3\in \mathbb Z\}=\mathcal L_{\mathrm{BCC}}(4\pi/a_1, 4\pi/a_2, 4\pi/a_3)
\end{equation}
Here $\mathcal L^*$ denotes the reciprocal of the lattice $\mathcal L$.

For an arbitrary lattice $\mathcal L$, suppose that $\mathcal M$ is the reciprocal lattice of the lattice $\mathcal L$, i.e., for every $\mathbf R\in\mathcal L$ and $\mathbf G\in\mathcal M$, $e^{i(\mathbf G\cdot\mathbf R)}=1$. This directly implies that for arbitrary $\mathbf G\in\mathcal M$, $e^{i(\mathbf R\cdot \mathbf G)}=e^{i(\mathbf G\cdot \mathbf R)}=1$ for every $\mathbf R\in\mathcal L$, which means that $\mathcal L$ is the reciprocal of $\mathcal M$. Therefore, $\mathcal L^*=\mathcal M$ if and only if $\mathcal M^*=\mathcal L$. 

Hence,
\begin{equation}
    \mathcal L_{\mathrm{BCC}}^*(4\pi/a_1, 4\pi/a_2, 4\pi/a_3)=\mathcal L_{\mathrm{FCC}}(a_1, a_2, a_3).
\end{equation}
by exchanging $a_1\leftrightarrow 4\pi/a_1$, $a_2\leftrightarrow 4\pi/a_2$, and $a_3\leftrightarrow 4\pi/a_3$,
\begin{equation}
    \mathcal L_{\mathrm{BCC}}^*(a_1, a_2, a_3)=\mathcal L_{\mathrm{FCC}}(4\pi/a_1, 4\pi/a_2, 4\pi/a_3).
\end{equation}

To sum up,

(a)
\begin{equation}
    \mathcal L_{\mathrm{BCC}}^*(a, a, a)=\mathcal L_{\mathrm{FCC}}(4\pi/a, 4\pi/a, 4\pi/a)
\end{equation}

(b)
\begin{equation}
    \mathcal L_{\mathrm{FCC}}^*(a, a, a)=\mathcal L_{\mathrm{BCC}}(4\pi/a, 4\pi/a, 4\pi/a).
\end{equation}

(c)\\
\begin{equation}
    \mathcal L_{\mathrm{FCC}}^*(a_1, a_2, a_3)=\mathcal L_{\mathrm{BCC}}(4\pi/a_1, 4\pi/a_2, 4\pi/a_3)
\end{equation}




\section{13.6 Brillouin Zones}

(a) For the cubic lattice $\mathcal L_{\text{cubic}}(a)=\{a\mathbf i:\mathbf i\in\mathbb Z^3\}$, the reciprocal lattice is $\mathcal L_{\text{cubic}}^*(a)=\{(2\pi/a)\mathbf i:\mathbf i\in\mathbb Z^3\}=\mathcal L_{\text{cubic}}(2\pi /a)$. The first Brillouin zone is $\mathrm{BZ}_1=[-\pi/a, \pi/a]^3$. 

For $\mathbf k=(k_x, k_y, k_z)\in\mathbb R^3$ and $\mathbf k'\in\mathrm{BZ}_1$,  they are equivalent if and only if there exists a reciprocal lattice vector $\mathbf G\in\mathcal L_{\text{cubic}}^*(a)$, such that $\mathbf k=\mathbf k'+ \mathbf G$. Let $f:\mathbb R\rightarrow [-\pi/a, \pi/a)$ such be defined by
\begin{equation}
    f(k)=x-\frac{2\pi}{a}\floor\left(\frac{a}{2\pi}k + \frac{1}{2}\right)
\end{equation}
Let $\mathbf k=(k_x, k_y, k_z)\in\mathbb R^3$, then $\mathbf k'=(f(k_x), f(k_y), f(k_z))\in\mathrm{BZ}_1$ , and they are equivalent since $\mathbf k=\mathbf k'+\mathbf G$ where $\mathbf G=(2\pi/a)(\floor((a/2\pi) k_x + 1/2), \floor((a/2\pi) k_y + 1/2), \floor((a/2\pi) k_z + 1/2))\in \mathcal L_{\text{cubic}}^*(a)$

(b) Let $\mathcal L_{\text{tri}}(a)=\{n_1\mathbf a_1+n_2\mathbf a_2:n_1, n_2\in\mathbb Z\}$, where $\mathbf a_1=a\mathbf e_1$ and $\mathbf a_2=\frac{a}{2}\mathbf e_1+\frac{\sqrt{3}a}{2}\mathbf e_2$. The reciprocal lattice is $\mathcal L_{\text{tri}}(a)=\{n_1\mathbf a_1^*+n_2\mathbf a_2^*:n_1, n_2\in\mathbb Z\}$, where $\mathbf a_1^*=\frac{2\pi}{a}\mathbf e_1-\frac{2\pi}{a\sqrt{3}}\mathbf e_2$ and $\mathbf a^*_2=\frac{4\pi}{a\sqrt{3}}\mathbf e_2$. Let $\mathbf a_3^*=\mathbf a_1^*+\mathbf a_2^*$. Then, the first Brillouin zone is the hexagon
\begin{equation}
    \mathrm{BZ}_1=\left\{\mathbf k\in\mathbb R^2:\frac{\mathbf a_{j}^*\cdot \mathbf k}{\|\mathbf a_j^*\|}\le\frac{\|a_j^*\|}{2}, j=1,2,3\right\}
\end{equation}







\section{14.4 Neutron Scattering}

Since a hydrogen atom has only one electron, the response to electromagnetic waves is much less than other atoms, including sodium. Hence, the signal-to-noise ratio is not high enough for the hydrogen, X-ray scattering experiment is not a good method for positioning hydrogen atoms. For the neutron scattering, although hydrogen has a smaller cross section compared to the others,  

By the Fermi golden rule, the transition rate is
\begin{align}
    \Gamma_{\mathbf k'\leftarrow \mathbf k} 
    &= \frac{2\pi}{\hbar}|\bra{\mathbf k'}V\ket{\mathbf k}|^2\delta(E(\mathbf k')-E(\mathbf k)) \\
    &= \frac{\pi\hbar}{m}|\bra{\mathbf k'}V\ket{\mathbf k}|^2\delta(\|\mathbf k'\|^2-\|\mathbf k\|^2)
\end{align}
Using the box normalization,
\begin{align}
    \Gamma_{\mathbf k'\leftarrow \mathbf k} 
    &= \frac{\pi\hbar}{m}\left(\frac{1}{L^3}\int d\mathbf x\, e^{i(\mathbf k-\mathbf k')\cdot\mathbf x} V(\mathbf x)\right)^2\delta(\|\mathbf k'\|^2-\|\mathbf k\|^2) \\
    &= \frac{\pi\hbar}{m}\left(\frac{1}{L^3}\sum_{\mathbf R}\int_{\text{unit cell}} d\mathbf x\, e^{i(\mathbf k-\mathbf k')\cdot(\mathbf x+\mathbf R)} V(\mathbf x+\mathbf R)\right)^2\delta(\|\mathbf k'\|^2-\|\mathbf k\|^2) \\
    &= \frac{\pi\hbar}{m}\left(\frac{1}{L^3} \sum_{\mathbf R}e^{i(\mathbf k-\mathbf k')\cdot\mathbf R}\int_{\text{unit cell}} d\mathbf x\, e^{i(\mathbf k-\mathbf k')\cdot\mathbf x} V(\mathbf x)\right)^2\delta(\|\mathbf k'\|^2-\|\mathbf k\|^2)
\end{align}
Where $\frac{1}{L^3} \sum e^{i(\mathbf k-\mathbf k')\cdot\mathbf R}$ is the zero unless the $\mathbf k-\mathbf k'$ is a reciprocal lattice vector, which is the Laue condition. And the integral, the second term, is the structure factor $S(\mathbf G)=\int_{\text{unit cell}} d\mathbf x\, e^{i\mathbf G\cdot\mathbf x}V(\mathbf x)$. Whatever unit cell we take, the above formula holds. Hence, take a conventional unit cell for convenience.

Assume the potential is constructed by the two types of delta potentials: 
\begin{equation}
    V(\mathbf x)=f_{\mathrm{Na}}\sum_{\mathbf x'\in\mathcal B_{\mathrm{Na}}}\delta(\mathbf x-\mathbf x')+f_{\mathrm{H}}\sum_{\mathbf x'\in\mathcal B_{\mathrm{H}}}\delta(\mathbf x-\mathbf x')
\end{equation}
Then the structure factor is
\begin{align}
    S(\mathbf G)
    &=\int_{\text{unit cell}}d\mathbf x e^{i\mathbf G\cdot\mathbf x}V(\mathbf x) \\
    &=\int_{\text{unit cell}}d\mathbf x e^{i\mathbf G\cdot\mathbf x}\left[f_{\mathrm{Na}}\sum_{\mathbf x'\in\mathcal B_{\mathrm{Na}}}\delta(\mathbf x-\mathbf x')+f_{\mathrm{H}}\sum_{\mathbf x'\in\mathcal B_{\mathrm{H}}}\delta(\mathbf x-\mathbf x')\right]\\
    &=f_{\mathrm{Na}}\sum_{\mathbf x\in\mathcal B_{\mathrm{Na}}}e^{i\mathbf G\cdot\mathbf x}+f_{\mathrm{H}}\sum_{\mathbf x\in\mathcal B_{\mathrm{H}}}e^{i\mathbf G\cdot\mathbf x} \\
    &=(f_{\mathrm{Na}} + f_{\mathrm{H}}e^{i\mathbf G\cdot \mathbf y})\sum_{\mathbf x\in\mathcal B_{\mathrm{FCC}}}e^{i\mathbf G\cdot\mathbf x}
\end{align}

(i) for the sodium chloride structure, ($\mathbf y=[1/2, 1/2, 1/2]$):
\begin{equation}
    S_{hkl}=(f_{\mathrm{Na}} + f_{\mathrm{H}}e^{\pi i(h+k+l)})(1+e^{i\pi(h+k)}+e^{i\pi(k+l)}+e^{i\pi(l+h)})\\
\end{equation}
\begin{align}
    S_{111}&=4(f_{\mathrm{Na}} - f_{\mathrm{H}})\\
    S_{200}&=4(f_{\mathrm{Na}} + f_{\mathrm{H}})\\
\end{align}
since $f_\mathrm H <0< f_{\mathrm{Na}}$, $|S_{200}|<|S_{111}|$ in this case. Since $\Gamma\propto |S|^2$, the calculated structure factor is consistent with the experimental result.


(ii) for the zinc blend structure, ($\mathbf y=[1/4, 1/4, 1/4]$):
\begin{equation}
    S_{hkl}=(f_{\mathrm{Na}} + f_{\mathrm{H}}e^{\pi i(h+k+l)/2})(1+e^{i\pi(h+k)}+e^{i\pi(k+l)}+e^{i\pi(l+h)})\\
\end{equation}
\begin{align}
    S_{111}&=4(f_{\mathrm{Na}} -i f_{\mathrm{H}})\\
    S_{200}&=4(f_{\mathrm{Na}} - f_{\mathrm{H}})\\
\end{align}

Since $|S_{200}|=4|f_{\mathrm{Na}} - f_{\mathrm{H}}|=4(|f_{\mathrm{Na}}|+|f_{\mathrm{H}}|)>4\sqrt{f_{\mathrm{Na}}^2 + f_{\mathrm{H}}^2}=|S_{111}|$, the peak indexed as (200) must be higher. Based on the experimental result, the sample is not the ZnS structure.  






\section{14.9 Form Factors}

















\end{document}
